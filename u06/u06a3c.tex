Da das Alphabet nur aus einem Zeichen besteht, korrespondiert eine Folge genau
dann, wenn das obere und untere Wort gleich lang sind.
Wir weisen jedem Domino die Differenz der Länge seines oberen und unteren Wortes
zu, also beispielsweise $d(\left[\frac{xx}{xxxxx}\right]) = -3$. Eine
korrespondierende Folge besteht also offensichtlich aus Dominos, deren
Differenzen summiert 0 ergeben.

Wir unterscheiden also zwei Fälle:

Wenn für alle Dominos $x$ $d(x) < 0$ oder für alle Dominos $x$ $d(x) > 0$ gilt,
verwerfen wir. Das PKP ist dann nicht lösbar, da jede Folge von Dominos eine
summierte Differenz von höchstens $-n$ im ersten Fall (mit $n =$ Länge der
Folge) bzw. $n$ im zweiten Fall hat.

Ansonsten akzeptieren wir. Wenn ein Domino $x$ mit $d(x) = 0$ existiert, dann
ist die Folge, die nur $x$ enthält, eine Lösung. Ansonsten lassen sich zwei
Dominos $x$ und $y$ wählen mit $d(x) < 0$ und $d(y) > 0$. Wenn man nun
$x$ $d(y)$-mal legt und danach $y$ $-d(x)$-mal, ist die summierte Differenz
dieser Folge $d(x)*d(y) + d(y)*(-d(x)) = d(x)*d(y) - d(y)*d(x) = 0$. Damit
korrespondiert diese Folge und es existiert eine Lösung für das PKP. Es ist also
lösbar.

Es lässt sich also nur durch Betrachten der Dominos feststellen, ob das PKP
lösbar ist. Damit ist diese PKP-Variante ebenfalls entscheidbar.\hfill$\square$
