\documentclass[a4paper,graphics,11pt]{article}

\usepackage[margin=1in]{geometry}
\usepackage[utf8]{inputenc}
\usepackage[T1]{fontenc}
\usepackage{lmodern}
\usepackage[ngerman]{babel}
\usepackage{amsmath, tabu}
\usepackage{amsthm}
\usepackage{amssymb}
\usepackage{algorithm}
\usepackage{algpseudocode}
\usepackage{mathtools}
\usepackage{setspace}
\usepackage{graphicx,color,curves,epsf,float,rotating}

\floatname{algorithm}{Algorithmus}

\newcommand\norm[1]{\left\lVert#1\right\rVert}
\newcommand\abs[1]{\left\vert#1\right\vert}

\newcommand\aufgabe[1]{\subsection*{Aufgabe #1}}
\newcommand\aufgabenteil[1]{\subsubsection*{#1}}



\pagestyle{empty}
\begin{document}
\noindent Gruppe \fbox{\textbf{17}}             \hfill Philipp Hochmann, 356148 \\
\noindent Berechenbarkeit \& Komplexität \hfill Felix Kiunke, 357322 \\

\begin{center}
	\LARGE{\textbf{Blatt ?}}                  % Nummer das Blattes, nicht vergessen zu ändern!
\end{center}
\begin{center}
\rule[0.1ex]{\textwidth}{1pt}
\end{center}


%%%%%%%%%%%%%%%%%%%%%%%%%%%%%%%%%%%%%%
%
%   Ab hier kommt der Text
%   Neue Aufgabe mit \aufgabe{}
%   Aufgabenteil mit \aufgabenteil{}
% 
%%%%%%%%%%%%%%%%%%%%%%%%%%%%%%%%%%%%%%

% \aufgabe{0.0}
% Hier Aufgabentext

\end{document}
% Nummer des Blattes angepasst?
