\aufgabenteil{(a)}
Idee: Unterprogrammtechnik, zuerst wird $w$ geschrieben, dann Kopf zurückgesetzt, dann $M$ ausgeführt. Formal:
\begin{align*}
	M &= (Q_M, \{0, 1\}, \Gamma, \delta_M, q_{0_M}, B, \bar{q})\\
	M_w^* &= (Q_w \cup Q_M, \{0, 1\}, \Gamma, \delta \cup \delta_M, q_0, B, \bar{q})
\end{align*}
wobei
\[
	Q_w = \{q_1 \dots q_n\} \text{ für } w = w_1w_2w_3\dots w_n
\]
und
\[
	\begin{tabular}{c|c}
		$\delta$ & B\\
		\hline
		$q_i$ & $(q_{i+1}, w_i, R)$
	\end{tabular}
	\ \forall i \in [1, n - 1] \subseteq \mathbb{N},
\]
\[
	\begin{tabular}{c|c|c|c}
		$\delta$ & 0 & 1 & B\\
		\hline
		$q_n$ & $(q_{0_M}, 0, N)$ & $(q_{0_M}, 1, N)$ & $(q_{0_M}, B, N)$
	\end{tabular}
\]


\aufgabenteil{(b)}
Idee: Gödelnummer um Übergänge ergänzen, die vorher $w$ schreiben.

$N$ sei eine Zweibandmaschine. Im ersten Band steht $\langle M_k \rangle$, im
zweiten $w$, was mit einer einfachen Vorverarbeitung des Eingabewortes geschehen
kann. Nun fängt $N$ an, $w$ Zeichen für Zeichen durchzugehen, und hinter dem
einleitenden $111$ der Original-Nummer auf Band 1 folgendes zu schreiben: bei
$w_i=X_j$, wobei $X_1=0, X_2=1, X_3=B$, schreibe $0^i10^j10^{i+1}10^j1000$. Dies
entspricht dem Gödel-Code von $q_i$ nach $q_{i+1}$, wobei das eingelesene
Zeichen geschrieben wird und der Lesekopf nach rechts bewegt wird. Diese Codes
müssen natürlich durch $11$ getrennt werden.

Ist $w$ auf Band 2 komplett
durchlaufen, müssen die alten Zustandsnummern der restlichen Gödelnummer erhöht
werden, weil sich durch das Hinzufügen neuer Zustände die Nummerierung geändert
hat. Dies macht man, indem man immer bei jedem Zustand so viele Nullen
hinzufügt, wie man vorher neue Zustände eingeführt hat ($n$ Stück plus die, die
man braucht, den Lesekopf wieder zurückzuschieben). Das Endergebnis steht dann
auf Band 1.
