Angenommen, $\mathcal{M}$ hält. Das heißt, das während der Ausführung keine
Konfiguration doppelt angenommen werden kann. Würde eine Konfiguration doppelt
vorkommen, würde $\mathcal{M}$ nicht halten (Analog Pumping Lemma). Insbesondere
wird die Anfangskonfiguration nicht wieder angenommen. Es gilt also, die Anzahl
der Konfigurationen zu berechnen, um eine obere Schranke der Schritte, die
$\mathcal{M}$ macht, angeben zu können.

Die Anzahl der Schritte kann genau mit der Formel
\[(|Q| - 1) \cdot |\Gamma|^{s(n)} \cdot s(n) + 1\]
berechnet werden.

Anzahl der Zustände ohne Endzustand $\times$ Anzahl der Bandbelegung $\times$
Anzahl der Lesekopfpositionen + 1 (für den Schritt in den Endzustand)
