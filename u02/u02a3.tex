Sei $L = \{w\#w\ |\ w \in \{0, 1\} ^{\ast}\}$ über $\Sigma = \{0, 1, \#\}$.

\aufgabenteil{(a)}

Wir konstruieren die Turingmaschine $M = (Q, \Sigma, \Gamma, B, q_0, \bar{q},
	\delta)$ mit
\begin{align*}
	\Sigma &= \{0, 1, \#\}, \\
	\Gamma &= \{0, 1, \#, B\}.
\end{align*}

Die Turingmaschine operiert folgendermaßen (High-Level-Beschreibung folgt weiter
unten):

\begin{enumerate}
	\item Lies und merke (im Zustand) das erste Zeichen $a$ und ersetze es dann
		durch $B$
	\item Gehe nach rechts, bis ein $\#$ oder ein $B$ erreicht wird. Wird zuerst
		ein	$B$ erreicht, schreibe $0$ und gehe in den Endzustand $\bar{q}$ über
	\item Gehe weiter nach rechts, bis zum ersten Zeichen, das kein $\#$ ist
	\item Wenn das Zeichen dem Zeichen $a$ nicht entspricht, schreibe $0$ und
		gehe in den Endzustand $\bar{q}$ über
	\item Wenn das Zeichen ein $b$ ist und $a$ ein $\#$ ist, schreibe $1$ und
		gehe in den Endzustand $\bar{q}$ über
	\item Ansonsten schreibe $\#$ und gehe einen Schritt nach rechts
	\item Lies und merke (im Zustand) das Zeichen $b$ und ersetze es durch $\#$
	\item Gehe nach links bis zum ersten $B$ und dann einen Schritt nach rechts
	\item Wenn das Zeichen dem gemerkten Zeichen $b$ nicht entspricht, schreibe
		$0$ und gehe in den Endzustand $\bar{q}$ über
	\item Ansonsten schreibe $B$ und gehe einen Schritt nach rechts.
	\item Fahre fort mit Schritt 1
\end{enumerate}

\textbf{High-Level-Beschreibung:} Die Maschine merkt sich immer ein Zeichen in
Form verschiedener Zustände und vergleicht es dann mit dem entsprechenden
Zeichen auf der anderen Seite des $\#$. Jedes gelesene Zeichen wird links durch
ein $B$ und rechts durch ein $\#$ ersetzt. Das erleichtert es, einen möglichst
kurzen Weg zurückzulegen. Außerdem wird, bevor der „Rückweg“ gegangen wird, das
jeweils nächste Zeichen (einen Schritt weiter rechts) eingelesen, um weitere
Strecken zu vermeiden. Wenn die Zeichen nicht übereinstimmen oder ein sonstiger
ungültiger Fall (z.B. kein $\#$ im Wort, unterschiedliche Längen des linken und
rechten Teils, \dots) auftritt, wird sofort $0$ geschrieben und abgebrochen,
also das Wort abgelehnt.

\textbf{Analyse:} Da die Turingmaschine nur auf dem Eingabewort operiert und
maximal eine Zelle rechts davon verwendet, ist der Speicherbedarf
$\mathcal{O}(n)$ mit $n = \text{Länge des Eingabewortes}$.

Um die Buchstaben links und rechts des $\#$ zu vergleichen, bewegt sich der
Lesekopf alle zwei Zeichen um $\left \lceil{\frac{n}{2}}\right \rceil$. Da die
konstanten Faktoren von der $\mathcal{O}$-Notation „verschluckt“ werden, diese
Bewegung des Lesekopfs aber proportional oft zur Zeichenanzahl ausgeführt wird,
ist der Zeitbedarf $\mathcal{O}(n^2)$.

\aufgabenteil{(b)}

Wir konstruieren eine 2-Band-Turingmaschine, die folgendermaßen funktioniert:

\begin{enumerate}
	\item Lies die Zeichen der Eingabe von links nach rechts ein, bis entweder
		$\#$ oder $B$ gelesen wird. Schreibe dabei jedes gelesene Zeichen (außer
		dem zuletzt gelesenen) von links nach rechts auf das zweite Band
	\item Wenn ein $B$ erreicht wurde, schreibe $0$ aufs erste Band und gehe in
		den Endzustand über
	\item Ansonsten gehe auf dem ersten Band einen Schritt nach rechts. Wenn
		dort ein $\#$ steht, ist die Eingabe ungültig. Schreibe dann $0$ und
		gehe in den Endzustand über
	\item Gehe auf dem zweiten Band nach links bis zum ersten $B$ und gehe dann
		wieder einen Schritt nach rechts
	\item Vergleiche die Zeichen auf beiden Bändern. Unterscheiden sie sich,
		schreibe sofort $0$ aufs erste Band und gehe in den Endzustand über.
		Sind beide Zeichen $B$, schreibe $1$ aufs erste Band und gehe in den
		Endzustand über (dann ist das Wort in $L$). Gehe einen Schritt nach
		rechts und wiederhole dies bis ein Endzustand erreicht wird
\end{enumerate}

\textbf{Analyse:} Die 2-Band-Maschine braucht nur den Platz, den das Eingabewort
braucht, und auf dem zweiten Band zusätzlich maximal ebensoviel Platz (wenn das
Wort kein $\#$ enthält). Der Speicherbedarf ist also $2n$ bzw. $\mathcal{O}(n)$
mit $n = \text{Länge des Eingabewortes}$.

Der Zeitbedarf für Schritt 1-3 ist im Worst Case $\mathcal{O}(n)$. Das
Zurückgehen in Schritt 4 ist ebenfalls $\mathcal{O}(n)$. Gleiches gilt für den
Vergleich in Schritt 5. Damit ist der Gesamtzeitbedarf $\mathcal{O}(n)$.
