An der 0. Position des Bandes, also ganz links vom Eingabewort, steht ein
Sentinel-Element, welches den Zugriff auf das Band regelt, es sozuagen nach
links absperrt. Man simuliert die linke Seite nun über das kartesische Produkt
(die resultierende TM ist eine Zweispurmaschine): Jede Zelle des Bandes rechts
vom Sentinel-Element enthalte ein Tupel. Links davon sind nur Blanks. Der erste
Eintrag ist das Element auf dem rechten Teil des Bandes, der zweite das Element,
was spiegelbildlich auf der linken Seite stehen würde. Die Turingmaschine
speichert über zwei Zustände, ob gerade auf dem ersten oder zweiten Element des
Tupels operiert wird. Wird der Lesekopf auf das Sentinel-Element geschoben, wird
er direkt wieder nach rechts auf das erste richtige Symbol geschoben. Dabei wird
in den jeweils anderen Zustand übergegangen, da nun die jeweils mit den anderen
Elementen der Tupel gearbeitet werden muss. Dadurch wird ein Übergang auf die
andere Bandhälfte simuliert.

Dabei entsteht kein Zeitverlust.
