\documentclass[a4paper,graphics,11pt]{article}

\usepackage[margin=1in]{geometry}
\usepackage[utf8]{inputenc}
\usepackage[T1]{fontenc}
\usepackage{lmodern}
\usepackage[ngerman]{babel}
\usepackage{amsmath, tabu}
\usepackage{amsthm}
\usepackage{amssymb}
\usepackage{algorithm}
\usepackage{algpseudocode}
\usepackage{mathtools}
\usepackage{setspace}
\usepackage{graphicx,color,curves,epsf,float,rotating}
\usepackage{alltt}
\usepackage{framed}
\setlength{\parindent}{0em}
\setlength{\parskip}{1em}

\floatname{algorithm}{Algorithmus}

\newcommand\norm[1]{\left\lVert#1\right\rVert}
\newcommand\abs[1]{\left\vert#1\right\vert}

\newcommand\aufgabe[1]{\subsection*{Aufgabe #1}}
\newcommand\aufgabenteil[1]{\subsubsection*{#1}}



\pagestyle{empty}
\begin{document}
\noindent Gruppe \fbox{\textbf{17}}             \hfill Philipp Hochmann, 356148 \\
\noindent Berechenbarkeit \& Komplexität \hfill Felix Kiunke, 357322 \\

\begin{center}
	\LARGE{\textbf{Blatt 11}}
\end{center}
\begin{center}
\rule[0.1ex]{\textwidth}{1pt}
\end{center}

% EXERCISES
\aufgabe{11.1}
Reduktionsabbildung: Jede Klausel  $(a \lor b \lor c)$ der Eingabe wird transformiert zu
\begin{center}
$(*)\qquad(a \lor b \lor \bar{x}) \land (a \lor c \lor \bar{y}) \land (c \lor b \lor \bar{z}) \land (x \lor y \lor z)$
\end{center}
wobei $x, y, z$ o.B.d.A. keine Literale in der Eingabe sind. Dies ist sicherlich polynomiell (linear in der Anz. der Klauseln).
\\ \ \\
Beweis. \textit{Korrektheit}\\ \ \\
$w \in \textsc{3SAT}\\ \implies$
In jeder Klausel $(a \lor b \lor c)$ von $w$ kommt mindestens ein wahres Literal vor\\$\implies$
Es können $x, y, z$ so gewählt werden, dass im Ausdruck $(*)$ in jeder Klausel mindestens ein Literal wahr und ein Literal falsch auftritt:\\
\null\quad (a) Sind mehr als ein Literal von $a, b, c$ wahr, so setze $x$ und $y$ jeweils auf $wahr$ und $z$ auf $falsch$\\
\null\quad (b) Ist nur eines wahr, so setze $x$ auf $falsch$, wenn es $c$ ist, $y$ auf $falsch$, wenn es $b$ ist, und $z$ auf $falsch$, wenn es $a$ ist.\\$\implies$
Jede Klausel von $(*)$ hat mind. ein wahres und mind. ein falsches Literal, insb. ist die Belegung dann erfüllend\\$\implies
w \in \textsc{Not-all-equal-SAT}$\\ \ \\
$w \notin \textsc{3SAT}\\ \implies$
In mind. einer Klausel $(a \lor b \lor c)$ von $w$ kommt kein wahres Literal vor\\$\implies$
$(*)$ könnte dann nur erfüllend gemacht werden, wenn $x, y, z$ jeweils auf $falsch$ gesetzt werden, dies ist wegen der letzten Klausel von (*) allerdings nicht erlaubt\\$implies$
Es gibt keine erfüllende Belegung von $(*)\\ \implies
w \notin \textsc{Not-all-equal-SAT}$\\ \ \\
Es gilt daher $\textsc{3SAT} \leq_p \textsc{Not-all-equal-SAT}$.
\aufgabe{11.2}
Algorithmus:\\
Solange der Graph noch Kanten enthält, wähle zufällig eine Kante, füge ihre Endknoten zum Ergebnis hinzu, und entferne alle zu diesen ihren Endknoten inzidenten Kanten.\\ \ \\
Beweis \textsc{2-Approximation}:\\
Ein Endknoten ist auf jeden Fall auch Teil der optimalen Lösung. Der Andere nicht notwenigerweise. Deshalb ist die berechnete Approximation schlimmstenfalls doppelt so groß.

\aufgabe{11.3}
Wenn \textsc{LongestPath} in $\P$ liegt, so können wir, um \textsc{HC} zu lösen,
auf die Eingabe von \textsc{HC} $G = (V, E)$ \textsc{LongestPath} mit
$b = |V| - 1$ ausführen. Wenn kein Pfad der Länge $b$ existiert, kann es auch
keinen Hamiltonkreis geben. Wenn einer existiert, so kann es einen Hamiltonkreis
geben, wenn eine Kante die zwei Endknoten des Pfades verbindet. Um das in
polynomieller Zeit zu testen, gehen wir folgendermaßen vor:

Für jede Kante $e \in E$: Entferne $e$ aus dem Graphen und füge zwei neue Knoten
$v_1$ und $v_2$ ein. Füge außerdem zwei neue Kanten $e_1$ und $e_2$ hinzu, die
zwischen den beiden Endknoten von $e$ und $v_1$ bzw. $v_2$ verlaufen. Führe
\textsc{LongestPath} auf $G' = (V \cup \{v_1, v_2\}, (E \setminus \{e\}) \cup \{e_1, e_2\})$
mit $b' = b + 2$ durch. Wenn nun ein Pfad der Länge $b'$ existiert, so muss er
die beiden Knoten $v_1$ und $v_2$ enthalten. Da diese Knoten nur über eine Kante
mit dem restlichen Graphen verbunden sind, müssen sie Endknoten sein. Also
existiert auch ein Pfad der Länge $b$ in $G$, dessen Endknoten die beiden Enden
von $e$ sind. Da wir die Kante $e$ in $G'$ entfernt haben, können wir sie auch
wieder zum Pfad hinzufügen, um ihn zu schließen und zu einem Hamiltonkreis zu
machen. Wir geben also Ja zurück. Wenn kein Pfad der Länge $b'$ existiert,
fahren wir mit der nächsten Kante fort. Wenn wir alle Kanten abgearbeiten haben,
geben wir Nein zurück, dann existiert kein Hamiltonkreis in $G$.

Da wir \textsc{LongestPath} höchstens $|E| + 1$-mal durchführen und alle anderen
Schritte offensichtlich auch polynomiell sind, können wir \textsc{HC} in
Polynomialzeit berechnen, wenn \textsc{LongestPath} in $\P$ liegt. Da \textsc{HC}
in $\NP$ liegt, kann \textsc{LongestPath} also auch nicht in $\P$ liegen (unter
der Annahme, dass $\P \neq \NP$). \hfill\(\square\)


\end{document}
