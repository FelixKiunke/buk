Wenn \textsc{LongestPath} in $\P$ liegt, so können wir, um \textsc{HC} zu lösen,
auf die Eingabe von \textsc{HC} $G = (V, E)$ \textsc{LongestPath} mit
$b = |V| - 1$ ausführen. Wenn kein Pfad der Länge $b$ existiert, kann es auch
keinen Hamiltonkreis geben. Wenn einer existiert, so kann es einen Hamiltonkreis
geben, wenn eine Kante die zwei Endknoten des Pfades verbindet. Um das in
polynomieller Zeit zu testen, gehen wir folgendermaßen vor:

Für jede Kante $e \in E$: Entferne $e$ aus dem Graphen und füge zwei neue Knoten
$v_1$ und $v_2$ ein. Füge außerdem zwei neue Kanten $e_1$ und $e_2$ hinzu, die
zwischen den beiden Endknoten von $e$ und $v_1$ bzw. $v_2$ verlaufen. Führe
\textsc{LongestPath} auf $G' = (V \cup \{v_1, v_2\}, (E \setminus \{e\}) \cup \{e_1, e_2\})$
mit $b' = b + 2$ durch. Wenn nun ein Pfad der Länge $b'$ existiert, so muss er
die beiden Knoten $v_1$ und $v_2$ enthalten. Da diese Knoten nur über eine Kante
mit dem restlichen Graphen verbunden sind, müssen sie Endknoten sein. Also
existiert auch ein Pfad der Länge $b$ in $G$, dessen Endknoten die beiden Enden
von $e$ sind. Da wir die Kante $e$ in $G'$ entfernt haben, können wir sie auch
wieder zum Pfad hinzufügen, um ihn zu schließen und zu einem Hamiltonkreis zu
machen. Wir geben also Ja zurück. Wenn kein Pfad der Länge $b'$ existiert,
fahren wir mit der nächsten Kante fort. Wenn wir alle Kanten abgearbeiten haben,
geben wir Nein zurück, dann existiert kein Hamiltonkreis in $G$.

Da wir \textsc{LongestPath} höchstens $|E| + 1$-mal durchführen und alle anderen
Schritte offensichtlich auch polynomiell sind, können wir \textsc{HC} in
Polynomialzeit berechnen, wenn \textsc{LongestPath} in $\P$ liegt. Da \textsc{HC}
in $\NP$ liegt, kann \textsc{LongestPath} also auch nicht in $\P$ liegen (unter
der Annahme, dass $\P \neq \NP$). \hfill\(\square\)
