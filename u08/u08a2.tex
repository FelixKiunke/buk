\aufgabenteil{(a)}
Der linke Graph ist in 2-COLORABILITY enthalten. Eine Lösung ist: \\
Schwarz: 3, 4, 6 \\
Weiß: 1, 2, 5, 7.

Der rechte Graph ist nicht enthalten. Dieser Graph enthält einen Kreis mit
ungerader Knotenanzahl (1, 3, 2, 4, 7). Dieser Kreis ist offensichtlich nicht
2-färbbar, also ist der gesamte Graph ebenfalls nicht färbbar.

\aufgabenteil{(b)}
Wir entwerfen folgenden Algorithmus: 

\begin{itemize}
\item Starte an einem beliebigen ungefärbten Knoten und färbe diesen weiß.
      Merke, dass der zuletzt eingefärbte Knoten weiß ist.
\item Wähle einen beliebigen Nachbarknoten $K$ des gewählten Knotens. Hat dieser
      dieselbe Farbe, verwirf, denn es gibt offensichtlich keine 2-Färbung.
	  Hat dieser eine andere Farbe, fahre mit einem anderen Nachbarn (sofern
	  vorhanden) fort. Ist $K$ noch ungefärbt, fahre mit $K$ fort.
\item Färbe $K$ mit dem „Gegenteil“ der gemerkten Farbe und merke die für $K$
      verwendete Farbe.
\end{itemize}

Diese Tiefensuche muss solange durchgeführt werden, wie es noch ungefärbte
Knoten gibt. Wenn alle Knoten eingefärbt sind, ist dies eine gültige 2-Färbung
und die Eingabe kann akzeptiert werden.

\textbf{Korrektheit:} Bei Akzeptanz ist die Korrektheit klar: Jeder neu
eingefärbte Knoten ist korrekt eingefärbt, da für alle Nachbarn geprüft wird, ob
diese schon die selbe Farbe haben und ansonsten verworfen wird. Außerdem werden
die noch nicht eingefärbten Nachbarknoten in einer anderen Farbe eingefärbt und
„passen“ somit auch.

Die Einfärbung im Algorithmus ist bis auf Umkehrung der Farben eindeutig. In
jedem Schritt ist der bisher eingefärbte Teilgraph gültig 2-gefärbt. Wenn nun
zwei Nachbarn in Schritt 2 die gleiche Farbe haben, sind sie (da eine
Tiefensuche durchgeführt wird) auch anderweitig (über andere Knoten) verbunden.
Das bedeutet, dass die Farbe dieser beiden bereits eingefärbten Knoten ebenfalls
bis auf Umkehrung eindeutig ist. Damit gibt es also keine andere gültige
Färbung, durch die die beiden Nachbarn nicht die gleiche Färbung erreichen. Also
ist der Algorithmus auch im „Verwerf-Fall“ korrekt

\textbf{Laufzeit:} Die Tiefensuche ist in $\mathcal{O}(|V| + |E|)$ durchführbar
mit $V =$ Knotenmenge und $E =$ Kantenmenge. Da die Kantenmenge durch die
Knotenmenge beschränkt ist, ist die Laufzeit also (grob abgeschätzt)
$\mathcal{O}(|V|^3)$. Die Überprüfung der Nachbarn ist in $\mathcal{O}(|V|)$,
also liegt die Gesamtlaufzeit in $\mathcal{O}(|V|^4)$ und damit in $P$
