Diese Aussage trifft zu, da die Sprache $A_{LOOP}$ entscheidbar ist. Sie ist insbesondere nicht schwieriger, als das Halteproblem. Eine Reduktion sähe so aus, dass  eine Abbildung $\langle P \rangle$ simuliert. LOOP-Programme haben eine feste Laufzeit und es ist daher entscheidbar, ob bei Eingabe 0 das Ergebnis 1 ist. Wenn ja, wird auf $\langle M_1 \rangle$ abgebildet, wenn nicht, auf $\langle M_2 \rangle$, wobei $M_1$ immer hält, und $M_2$ nie.