Um die Abzählbarkeit von $\mathbb{N}^*$ zu zeigen, zeigen wir zunächst, dass für
eine gegebene Länge $n$ die Menge der Wörter dieser Länge $\mathbb{N}^n$
abzählbar ist.

Für $n=0$ ist die Menge der Wörter offensichtlich $\{\varepsilon\}$.

Für $n=1$ ist die Menge der Wörter einfach $\mathbb{N}$. Per Induktion
konstruieren wir für alle $n \ge 1$ die entsprechenden Wortmengen. Dabei
verwenden wir das Diagonalverfahren, das auch zum Beweis der Abzählbarkeit der
rationalen Zahlen verwendet werden kann. Wie bei den rationalen Zahlen
konstruieren wir also eine Aufzählung auf das kartesische Produkt
$\mathbb{N}\times\mathbb{N}$. Der erste Eintrag der entstehenden Tupel
$(x, i)$ ist jeweils eine neue Stelle, während der zweite Eintrag für das
$i$-te Wort der Menge $\mathbb{N}^{n-1}$ steht. Das entstehende Wort ist also
die Konkatination von $x$ und dem $i$-ten nächstkürzeren Wort. Das funktioniert,
da $\mathbb{N}^{n-1}$ wegen der Induktion ebenfalls abzählbar ist. Die genaue
Abzählreihenfolge ist dabei beliebig.

Um zu zeigen, dass die Vereinigung dieser (jeweils abzählbaren) Mengen ebenfalls
abzählbar ist, verwenden wir erneut das Diagonalverfahren und konstruieren
wieder eine Aufzählung für $\mathbb{N}\times\mathbb{N}$. Hierbei steht der erste
Eintrag der Tupel der Form $(n, i)$ dieses kartesischen Produktes für die 
Länge $n$ der Wörter. Der zweite Eintrag steht für das $i$-te Wort in
$\mathbb{N}^n$, was ja wie oben gezeigt abzählbar ist. Eine Ausnahme muss bei
$n = 0$ gemacht werden, da diese $\mathbb{N}^0$ ja $1$-elementig ist. Daher
steht dort jedes $i$ für das leere Wort. Die genaue Zählreihenfolge dieser
Konstruktion ist auch hier egal.

Damit haben wir eine Art der Abzählung von $\mathbb{N}^*$ definiert. Also ist
diese Menge abzählbar. \hfill$\square$
