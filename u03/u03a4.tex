\aufgabenteil{(a)}
Diese Sprache ist entscheidbar.

Um sie zu entscheiden, konstruiert man eine modifizierte Universelle
Turingmaschine, die $M$ auf der Eingabe $w$ ausführt und auf einem weiteren Band
die Schritte mitzählt (zum Beispiel indem dieser Lesekopf bei jedem Schritt um
eine Stelle nach rechts bewegt wird).
Wenn der Endzustand von $M$ erreicht wird, schreibt die Universelle Maschine $1$
aufs erste Band und geht ihrerseits in den Endzustand über. Wenn vorher der 42.
Schritt gezählt wird, schreibt sie $0$ aufs erste Band und endet. Damit ist die
Sprache in endlich vielen Schritten entschieden. \hfill$\square$

\aufgabenteil{(b)}
Diese Sprache ist nicht entscheidbar.

Wenn es eine Maschine gäbe, die entscheidet, ob $M$ nach mindestens 42 Schritten
auf der Eingabe hält, so könnte diese auch das allgemeine Halteproblem
entscheiden. Dazu würde man zur zu untersuchenden Maschine $M'$ eine weitere
Maschine $M''$ konstruieren, die erst 42 „no-op“-Schritte ausführt und
anschließend in den Startzustand von $M'$ übergeht. Damit könnte das allgemeine
Halteproblem gelöst werden.

Da das Halteproblem nach VL nicht lösbar ist, ist dies ein Widerspruch. Somit
kann eine solche Maschine, die $H_{\geq 42}$ entscheidet, nicht existieren.
\hfill$\square$
