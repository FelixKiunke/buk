Gegeben sei die Sprache
\[
A_{62} = \{\langle M \rangle \mid M \text{ akzeptiert mindestens zwei Wörter der
	Länge höchstens } 62 \}
\]

Diese Sprache ist nicht entscheidbar, also \textbf{nicht rekursiv}.

\begin{proof}
	(Mit Satz von Rice). Sei 
\[
\mathcal{S} = \{f_M \mid \exists (w_1, w_2) \in \{0,1\}^* \times \{0,1\}^*,
	|w_1| \leq 62, |w_2| \leq 62, w_1 \neq w_2 : f_M(w_1) = 1 \land f_M(w_2) = 1\}
\]

Dann ist $\mathcal{S}$ nicht leer, denn eine passende Funktion $f_M$ existiert,
zum Beispiel die Funktion zur Turingmaschine, die alle Wörter aus $\{0, 1\}^*$
akzeptiert. Außerdem ist $\mathcal{S} \neq \mathcal{R}$, da nicht alle
Turingmaschinen die gewünschte Bedingung erfüllen.

Damit ist
\begin{align*}
	L(\mathcal{S}) &= \{ \langle M \rangle \mid M
		\text{ berechnet eine Funktion aus }\mathcal{S}\} \\
	&= \{ \langle M \rangle \mid M
		\text{ akzeptiert mindestens zwei Wörter der Länge höchstens } 62 \} \\
	&= A_{62}
\end{align*}
und somit ist $A_{62}$ nach Satz von Rice nicht entscheidbar.
\end{proof}

Allerdings ist $A_{62}$ \textbf{rekursiv aufzählbar}.

\begin{proof}
Wir konstruieren einen Aufzähler $A$ zur TM $M$, der auf zwei Bändern für
$i = 1, 2, 3, \dots$ jeweils $i$ Schritte für alle Wörter
$\{w_1, \dots, w_{\text{min}\{i, 62\}}\}$ simuliert. Dabei sind die für die
beiden Bänder gewählten Eingabewörter aus dieser Menge unabhängig voneinander.
Die Kombinationen sind analog zum Diagonalverfahren abzählbar. Auf beiden
Bändern wird jeweils die Turingmaschine $M$ simuliert. Wenn in einem Schritt
beide Bänder akzeptieren, wird auf dem Drucker $\langle M \rangle$ ausgegeben.

Anschließend konstruieren wir einen weiteren Aufzähler $A'$, der für
$i' = 1, 2, 3, \dots$ jeweils $i'$ Schritte des obigen Aufzählers $A$ für alle
möglichen Gödelnummern der Länge $i'$ ausführt. Jede Ausgabe von $A$ wird auch
von $A'$ ausgegeben. Damit gibt $A'$ alle Gödelnummern aus der Sprache $A_{62}$
aus.

	Dieser Aufzähler gibt dann genau $A_{62}$ aus. Da die Konstruktion eines
	Aufzählers möglich ist, ist $A_{62}$ rekursiv aufzählbar.
\end{proof}
