$f$ verwirft, wenn die Eingabe $w = \langle M \rangle$ keine gültige Gödelnummer
ist. Sonst gibt $f$ die Gödelnummer $\langle M^* \rangle$ aus. $M^*$ hat
folgende Eigenschaften: wenn die Eingabe $w$ nicht mit $101$ beginnt, wird
verworfen. Sollte $w$ mit $101$ beginnen, wird $M$ auf dem leeren Wort
$\varepsilon$ simuliert. Mit jedem Schritt dieser Simulation wird das nächste
Wort von der Gestalt $u = 101x$ mit $x \in \Sigma^*$ abgezählt und akzeptiert,
wenn dieses die Eingabe ist. Wenn die Simulation von $M$ abgeschlossen ist wird
in einen nicht-terminierenden Kreislauf gegangen. Diese Funktion ist
\textit{trivialerweise} berechenbar.

\begin{proof}
	(Korrektheit)
\begin{align*}
	\langle M \rangle \notin \overline{H_{\varepsilon}}
	&\implies M \text{ hält auf } \varepsilon \\
	&\implies M^*  \text{ hält auf keinem $w$, welches mit 101 beginnt} \\
	&\implies f(\langle M \rangle) \notin L_{101}
	\\\\
	\langle M \rangle \in \overline{H_{\varepsilon}}
	&\implies \text{$M$ hält nicht auf $\varepsilon$} \\
	&\implies M^*  \text{ simuliert abzählbar-unendlich viele Schritte lang} \\
	&\implies M^*  \text{ akzeptiert jedes $w$, welches mit 101 beginnt.
		Insbesondere hält $M^*$ dann.} \\
	&\implies f(\langle M \rangle) \in L_{101}
\end{align*}
\end{proof}
