\aufgabenteil{(a)}
\begin{align*}
	&L_1 \leq L_2 \wedge L_2 \leq L_3 \\
	\iff \quad &(\exists f: x \in L_1 \iff f(x) \in L_2) \land
		(\exists g: x \in L_2 \iff g(x) \in L_3)\\
\end{align*}

Also existiert eine solche Funktion $h(x)=f(g(x))$, wenn $f$ und $g$ existieren,
für die gilt:
\[
x \in L_1 \iff f(x) \in L_2 \iff h(x) \in L_3
\]

\aufgabenteil{(b)}
zu zeigen: \quad$(L_1 \leq L_2) \implies (\overline{L_1} \leq \overline{L_2})$
\begin{align*}
	&L_1 \leq L_2\\
	\iff &(\exists f: x \in L_1 \iff f(x) \in L_2)\\
	\implies &(\exists f: \neg (x \in L_1) \iff \neg (f(x) \in L_2))\\
	\implies &(\exists f: x \notin L_1 \iff f(x) \notin L_2)\\
	\implies &(\exists f: x \in \overline{L_1} \iff f(x) \in \overline{L_2})\\
	\implies &\overline{L_1} \leq \overline{L_2}
\end{align*}
