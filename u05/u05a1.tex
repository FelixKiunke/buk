\aufgabenteil{(a)}
$
\begin{aligned}
&L_1 \leq L_2 \wedge L_2 \leq L_3\\
\Leftrightarrow \quad &(\exists f: x \in L_1 \Leftrightarrow f(x) \in L_2) \wedge (\exists g: x \in L_2 \Leftrightarrow g(x) \in L_3)\\
\end{aligned}
$
\\ \ \\
Also existiert eine solche Funktion $h(x)=f(g(x))$ wenn $f$ und $g$ existieren, für die gilt:\\
$ x \in L_1 \Leftrightarrow f(x) \in L_2 \Leftrightarrow h(x) \in L_3$

\aufgabenteil{(b)}
zu zeigen: \quad$(L_1 \leq L_2) \implies (\overline{L_1} \leq \overline{L_2})\\ \ \\
\begin{aligned}
& &&L_1 \leq L_2\\
&\Leftrightarrow &&\exists f: x \in L_1 \Leftrightarrow f(x) \in L_2\\
&\implies &&\exists f: \neg (x \in L_1) \Leftrightarrow \neg (f(x) \in L_2)\\
&\implies && \exists f: x \notin L_1 \Leftrightarrow f(x) \notin L_2\\
&\implies && \exists f: x \in \overline{L_1} \Leftrightarrow f(x) \in \overline{L_2}\\
&\implies && \overline{L_1} \leq \overline{L_2}
\end{aligned}$
