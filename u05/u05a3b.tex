Gegeben sei die Sprache
\[
	B_1 = \{ \langle M \rangle \mid M \text{ akzeptiert mindestens ein Wort} \}
\]
Diese Sprache ist ebenfalls \textbf{nicht rekursiv}.

\begin{proof} mit dem Satz von Rice.
Sei
\[
\mathcal{S} = \{f_M \mid \exists w \in \{0, 1\}^* : f_M(w) = 1\}
\]

Dann ist $\mathcal{S}$ nicht leer, denn eine passende Funktion $f_M$ existiert,
zum Beispiel die Funktion zur Turingmaschine, die alle Wörter aus $\{0, 1\}^*$
akzeptiert. Außerdem ist $\mathcal{S} \neq \mathcal{R}$, da z.B. die Funktion
zur Turingmaschine, die alle Wörter verwirft, nicht in $\mathcal{S}$ ist.

Damit ist
\begin{align*}
	L(\mathcal{S}) &= \{\langle M \rangle \mid M
		\text{ berechnet eine Funktion aus }\mathcal{S}\} \\
	&= \{\langle M \rangle \mid M \text{ akzeptiert mindestens ein Wort}\} \\
	&= B_1
\end{align*}
und somit ist $B_1$ nach Satz von Rice nicht entscheidbar.
\end{proof}

Allerdings ist auch $B_1$ \textbf{rekursiv aufzählbar}.

\begin{proof}
Wir konstruieren wieder einen Aufzähler, der für $i = 1, 2, 3, \dots$
alle Wörter $\{w_1, \dots, w_i\}$ für $i$ Schritte simuliert, wobei diese Wörter
als Gödelnummern mit anschließenden Eingabewörtern betrachtet werden und
ungültige Gödelnummern zuvor verworfen bzw. ignoriert werden. Wenn eine dieser
Simulationen akzeptiert, wird die entsprechende Gödelnummer auf dem Drucker
ausgegeben (wobei das dazugehörige Eingabewort natürlich nicht mit ausgegeben
wird).

Dieser Aufzähler gibt also alle Gödelnummern aus $B_1$ aus. Da ein Aufzähler
existiert, ist $B_1$ offentsichtlich rekursiv aufzählbar.
\end{proof}
