\documentclass[a4paper,graphics,11pt]{article}

\usepackage[margin=1in]{geometry}
\usepackage[utf8]{inputenc}
\usepackage[T1]{fontenc}
\usepackage{lmodern}
\usepackage[ngerman]{babel}
\usepackage{amsmath, tabu}
\usepackage{amsthm}
\usepackage{amssymb}
\usepackage{algorithm}
\usepackage{algpseudocode}
\usepackage{mathtools}
\usepackage{setspace}
\usepackage{graphicx,color,curves,epsf,float,rotating}
\usepackage{alltt}
\setlength{\parindent}{0em}
\setlength{\parskip}{1em}

\floatname{algorithm}{Algorithmus}

\newcommand\norm[1]{\left\lVert#1\right\rVert}
\newcommand\abs[1]{\left\vert#1\right\vert}

\newcommand\aufgabe[1]{\subsection*{Aufgabe #1}}
\newcommand\aufgabenteil[1]{\subsubsection*{#1}}



\pagestyle{empty}
\begin{document}
\noindent Gruppe \fbox{\textbf{17}}             \hfill Philipp Hochmann, 356148 \\
\noindent Berechenbarkeit \& Komplexität \hfill Felix Kiunke, 357322 \\

\begin{center}
	\LARGE{\textbf{Blatt 5}}
\end{center}
\begin{center}
\rule[0.1ex]{\textwidth}{1pt}
\end{center}

% EXERCISES
\aufgabe{5.1}
\aufgabenteil{(a)}
$
\begin{aligned}
&L_1 \leq L_2 \wedge L_2 \leq L_3\\
\Leftrightarrow \quad &(\exists f: x \in L_1 \Leftrightarrow f(x) \in L_2) \wedge (\exists g: x \in L_2 \Leftrightarrow g(x) \in L_3)\\
\end{aligned}
$
\\ \ \\
Also existiert eine solche Funktion $h(x)=f(g(x))$ wenn $f$ und $g$ existieren, für die gilt:\\
$ x \in L_1 \Leftrightarrow f(x) \in L_2 \Leftrightarrow h(x) \in L_3$

\aufgabenteil{(b)}
zu zeigen: \quad$(L_1 \leq L_2) \implies (\overline{L_1} \leq \overline{L_2})\\ \ \\
\begin{aligned}
& &&L_1 \leq L_2\\
&\Leftrightarrow &&\exists f: x \in L_1 \Leftrightarrow f(x) \in L_2\\
&\implies &&\exists f: \neg (x \in L_1) \Leftrightarrow \neg (f(x) \in L_2)\\
&\implies && \exists f: x \notin L_1 \Leftrightarrow f(x) \notin L_2\\
&\implies && \exists f: x \in \overline{L_1} \Leftrightarrow f(x) \in \overline{L_2}\\
&\implies && \overline{L_1} \leq \overline{L_2}
\end{aligned}$

\aufgabe{5.2}
Zu widerlegen:
\begin{equation}\label{eq:5.2(1)}
	\overline{\mathbf{H}_\varepsilon} \leq \mathbf{H}_\varepsilon
\end{equation}
\begin{equation}\label{eq:5.2(2)}
	\mathbf{H}_\varepsilon \leq \overline{\mathbf{H}_\varepsilon}
\end{equation}
\begin{proof}[\unskip\nopunct]
Aus der Vorlesung ist bekannt, dass $\mathbf{H}_\varepsilon$ rekursiv aufzählbar
ist und dass $\overline{\mathbf{H}_\varepsilon}$ \textit{nicht} rekursiv
aufzählbar ist. Des Weiteren wurde gezeigt, dass
\[
	L_1 \leq L_2 \text{ und } L_2 \text{ rekursiv aufzählbar} \implies L_1
	\text{ rekursiv aufzählbar.}
\]
Damit ist die \eqref{eq:5.2(1)} widerlegt: $\textbf{H}_\varepsilon$ ist rekursiv
aufzählbar, $\overline{\textbf{H}_\varepsilon}$ allerdings nicht, was einen
Widerspruch zu dieser Aussage darstellen würde. Damit kann \eqref{eq:5.2(1)}
nicht gelten.
\end{proof}

Keine Ahnung, wie man die \eqref{eq:5.2(2)} widerlegt :(

\aufgabe{5.3}
\aufgabenteil{(a)}
Gegeben sei die Sprache
\[
A_{62} = \{\langle M \rangle \mid M \text{ akzeptiert mindestens zwei Wörter der
	Länge höchstens } 62 \}
\]

Diese Sprache ist nicht entscheidbar, also \textbf{nicht rekursiv}.

\begin{proof}
	(Mit Satz von Rice). Sei 
\[
\mathcal{S} = \{f_M \mid \exists (w_1, w_2) \in \{0,1\}^* \times \{0,1\}^*,
	|w_1| \leq 62, |w_2| \leq 62, w_1 \neq w_2 : f_M(w_1) = 1 \land f_M(w_2) = 1\}
\]

Dann ist $\mathcal{S}$ nicht leer, denn eine passende Funktion $f_M$ existiert,
zum Beispiel die Funktion zur Turingmaschine, die alle Wörter aus $\{0, 1\}^*$
akzeptiert. Außerdem ist $\mathcal{S} \neq \mathcal{R}$, da nicht alle
Turingmaschinen die gewünschte Bedingung erfüllen.

Damit ist
\begin{align*}
	L(\mathcal{S}) &= \{ \langle M \rangle \mid M
		\text{ berechnet eine Funktion aus }\mathcal{S}\} \\
	&= \{ \langle M \rangle \mid M
		\text{ akzeptiert mindestens zwei Wörter der Länge höchstens } 62 \} \\
	&= A_{62}
\end{align*}
und somit ist $A_{62}$ nach Satz von Rice nicht entscheidbar.
\end{proof}

Allerdings ist $A_{62}$ \textbf{rekursiv aufzählbar}.

\begin{proof}
Wir konstruieren einen Aufzähler $A$ zur TM $M$, der auf zwei Bändern für
$i = 1, 2, 3, \dots$ jeweils $i$ Schritte für alle Wörter
$\{w_1, \dots, w_{\text{min}\{i, 62\}}\}$ simuliert. Dabei sind die für die
beiden Bänder gewählten Eingabewörter aus dieser Menge unabhängig voneinander.
Die Kombinationen sind analog zum Diagonalverfahren abzählbar. Auf beiden
Bändern wird jeweils die Turingmaschine $M$ simuliert. Wenn in einem Schritt
beide Bänder akzeptieren, wird auf dem Drucker $\langle M \rangle$ ausgegeben.

Anschließend konstruieren wir einen weiteren Aufzähler $A'$, der für
$i' = 1, 2, 3, \dots$ jeweils $i'$ Schritte des obigen Aufzählers $A$ für alle
möglichen Gödelnummern der Länge $i'$ ausführt. Jede Ausgabe von $A$ wird auch
von $A'$ ausgegeben. Damit gibt $A'$ alle Gödelnummern aus der Sprache $A_{62}$
aus.

	Dieser Aufzähler gibt dann genau $A_{62}$ aus. Da die Konstruktion eines
	Aufzählers möglich ist, ist $A_{62}$ rekursiv aufzählbar.
\end{proof}


\aufgabenteil{(b)}
Gegeben sei die Sprache
\[
	B_1 = \{ \langle M \rangle \mid M \text{ akzeptiert mindestens ein Wort} \}
\]
Diese Sprache ist ebenfalls \textbf{nicht rekursiv}.

\begin{proof} mit dem Satz von Rice.
Sei
\[
\mathcal{S} = \{f_M \mid \exists w \in \{0, 1\}^* : f_M(w) = 1\}
\]

Dann ist $\mathcal{S}$ nicht leer, denn eine passende Funktion $f_M$ existiert,
zum Beispiel die Funktion zur Turingmaschine, die alle Wörter aus $\{0, 1\}^*$
akzeptiert. Außerdem ist $\mathcal{S} \neq \mathcal{R}$, da z.B. die Funktion
zur Turingmaschine, die alle Wörter verwirft, nicht in $\mathcal{S}$ ist.

Damit ist
\begin{align*}
	L(\mathcal{S}) &= \{\langle M \rangle \mid M
		\text{ berechnet eine Funktion aus }\mathcal{S}\} \\
	&= \{\langle M \rangle \mid M \text{ akzeptiert mindestens ein Wort}\} \\
	&= B_1
\end{align*}
und somit ist $B_1$ nach Satz von Rice nicht entscheidbar.
\end{proof}

Allerdings ist auch $B_1$ \textbf{rekursiv aufzählbar}.

\begin{proof}
Wir konstruieren wieder einen Aufzähler, der für $i = 1, 2, 3, \dots$
alle Wörter $\{w_1, \dots, w_i\}$ für $i$ Schritte simuliert, wobei diese Wörter
als Gödelnummern mit anschließenden Eingabewörtern betrachtet werden und
ungültige Gödelnummern zuvor verworfen bzw. ignoriert werden. Wenn eine dieser
Simulationen akzeptiert, wird die entsprechende Gödelnummer auf dem Drucker
ausgegeben (wobei das dazugehörige Eingabewort natürlich nicht mit ausgegeben
wird).

Dieser Aufzähler gibt also alle Gödelnummern aus $B_1$ aus. Da ein Aufzähler
existiert, ist $B_1$ offentsichtlich rekursiv aufzählbar.
\end{proof}



\aufgabe{5.4}
$f$ verwirft, wenn die Eingabe $w = \langle M \rangle$ keine gültige Gödelnummer ist. Sonst gibt $f$ die Gödelnummer $\langle M^* \rangle$ aus. $M^*$ hat folgende Eigenschaften: wenn die Eingabe $w$ nicht mit $101$ beginnt, wird verworfen. Sollte $w$ mit $101$ beginnen, wird $M$ auf dem leeren Wort $\varepsilon$ simuliert. Mit jedem Schritt dieser Simulation wird das nächste Wort von der Gestalt $u = 101x$ mit $x \in \Sigma^*$ abgezählt und akzeptiert, wenn dieses die Eingabe ist. Wenn die Simulation von $M$ abgeschlossen ist wird in einen nicht-terminierenden Kreislauf gegangen. Diese Funktion ist \textit{trivialerweise} berechenbar.\\ \ \\
Beweis der Korrektheit:\\
$\langle M \rangle \notin \overline{H_{\varepsilon}} \implies M \text{hält auf } \varepsilon \implies M^*  \text{ hält auf keinem $w$, welches mit 101 beginnt} \implies f(\langle M \rangle) \notin L_{101}$\\ \ \\
$\langle M \rangle \in \overline{H_{\varepsilon}} \implies \text{$M$ hält nicht auf $\varepsilon$} \implies M^*  \text{ simuliert abzählbar-unendlich viele Schritte lang} \implies M^*  \text{ akzeptiert jedes $w$, welches mit 101 beginnt. Insbesondere hält $M^*$ dann.} \implies f(\langle M \rangle) \in L_{101}$.

\end{document}
