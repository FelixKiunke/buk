\documentclass[11pt, oneside]{article}   	% use "amsart" instead of "article" for AMSLaTeX format
\usepackage[utf8x]{inputenc}
\usepackage{geometry}                		% See geometry.pdf to learn the layout options. There are lots.
\geometry{letterpaper}                   		% ... or a4paper or a5paper or ... 
%\geometry{landscape}                		% Activate for rotated page geometry
%\usepackage[parfill]{parskip}    		% Activate to begin paragraphs with an empty line rather than an indent
\usepackage{graphicx}				% Use pdf, png, jpg, or eps§ with pdflatex; use eps in DVI mode
								% TeX will automatically convert eps --> pdf in pdflatex		
\usepackage{amssymb}
\usepackage{amsmath}
\usepackage{array}
\newcolumntype{L}{>{$}l<{$}}

%SetFonts

%SetFonts


\title{BuK – Blatt 1 – Gruppe 17}
\author{Felix Kiunke, 357322\\Philipp Hochmann, 356148}
\date{\vspace{-5ex}}

\begin{document}

\maketitle

\section*{Aufgabe 1.1}
\subsection*{1.1 a)}
 
$\Sigma = \{ \#, 0, 1 \}$\\
$L_{Teilsumme} = \{ b\#M_1\#M_2\#...\#M_n | \exists T \subseteq M : \sum_{t \in T} t = b \}$\\
wobei $b$ und $M_i$ in Bin\"ardarstellung kodiert werden.
 
\subsection*{1.1 b)}
$\Sigma = \{ \#, \checkmark, X, 0, 1 \}$\\
$L_{Clique} = \{ b \# M | \exists $ Clique in durch M codierten Graph der Gr\"o\ss{}e von mind. b$ \}$\\
wobei b in Bin\"ardarstellung kodiert wird und M die Adjazenzmatrix als String kodiert bezeichnet.\\
Bsp.:
$
\begin{pmatrix}
X & \checkmark & X \\
\checkmark & X & \checkmark \\
X & \checkmark & X \\
\end{pmatrix}
$ wird zu $M=X \checkmark X \# \checkmark X \checkmark \# X \checkmark X$

\section*{Aufgabe 1.2}
Gegeben die Turingmaschine $M = (Q, \Sigma, \Gamma, B, q_0, \bar{q}, \delta)$ mit
$$
	Q = \{q_0, q_1, \bar{q}\},
	\Sigma = \{0, 1\},
	\Gamma = \{0, 1, B\},
$$
\begin{center}
\begin{tabular}{L|LLL}
	\delta & 0 & 1 & B \\
	\hline
	q_0 & (q_0, 0, R) & (q_0, 1, R) & (q_1, B, L) \\
	q_1 & (\bar{q}, 0, R) & (q_1, 1, L) & (q_0, B, R)
\end{tabular}
\end{center}
Dann werden mit der Eingabe $w = 110$ folgende Konfigurationen erreicht:
$$
q_0 110 \vdash 1 q_0 10 \vdash 11 q_0 0 \vdash 110 q_0 \vdash 11 q_1 0 \vdash 110 \bar{q}
$$

\section*{Aufgabe 1.3}
Gegeben die Turingmaschine $M = (Q, \Sigma, \Gamma, B, q_0, \bar{q}, \delta)$ mit
$$
	Q = \{q_0, q_1, q_2, q_3, q_4, \bar{q}\},
	\Sigma = \{0, 1\},
	\Gamma = \{0, 1, B\},
$$
\begin{center}
\begin{tabular}{L|LLL}
	\delta & 0 & 1 & B \\
	\hline
	q_0 & (q_1, 0, R) & (q_2, 1, R) & (\bar{q}, 0, N) \\
	q_1 & (q_1, 0, R) & (q_1, 1, R) & (q_3, B, L) \\
	q_2 & (q_2, 0, R) & (q_2, 1, R) & (q_4, B, L) \\
	q_3 & (\bar{q}, 0, N) & (\bar{q}, 1, N) & (\bar{q}, 0, N) \\
	q_4 & (\bar{q}, 1, N) & (\bar{q}, 0, N) & (\bar{q}, 0, N)
\end{tabular}
\end{center}
Wenn das erste Zeichen der Eingabe ungleich dem letzten Zeichen derselben ist, ist die Ausgabe von $M$ $1$, ansonsten $0$.\\\\
Von $M$ wird also die Funktion $f : \Sigma^* \rightarrow \{0, 1\}$ mit
\[
f(w) =
\begin{cases}
	0, &\text{falls } |w| < 2\\
	v \text{ xor } u, &\text{sonst, wobei } w = vw'u \text{ mit } v, u \in \{0, 1\}
\end{cases}
\]

\section*{Aufgabe 1.4}
$M=(\{q_0, q_1, q_2, q_e, \overline{q} \}, \{0, 1 \}, \{0, 1, B \}, B, q_0, \overline{q}, \delta)$
\\ \ \\
$\delta = $
\begin{tabular}{c|c|c|c}
& 0 & 1 & B\\
\hline
$q_0$ & $(q_0, 0, R)$ & $(q_0, 1, R)$ & $(q_1, B, L)$\\
$q_1$ & $(q_2, 0, L)$ & $(q_2, 1, L)$ & $(\overline{q}, B, N)$\\
$q_2$ & $(q_e, 1, N)$ & $(q_2, 0, L)$ & $(\overline{q}, 1, N)$\\
$q_e$ & $(q_e, 0, L)$ & $(q_e, 1, L)$ & $(\overline{q}, B, R)$\\
\end{tabular}
\\ \ \\
Zuerst wird in $q_0$ der Lesekopf von links nach rechts zum least significant Bit bewegt.\\
In $q_1$ wird geendet, wenn das aktuelle Symbol ein Blank ist, denn dann ist das Eingabewort $\epsilon$. Sonst wird der Lesekopf ein Symbol nach links verschoben, zum zweiten Bit von rechts, und in Zustand $q_2$ \"ubergegangen.\\
In $q_2$ f\"ahrt der Lesekopf von rechts nach links das Eingabewort ab und negiert die Bits (Addition von 1), bis kein \"Uberlauf mehr stattfindet. Ein Blank wird als 0 gewertet, f\"ur den Fall, dass die Aufgabe ein Symbol l\"anger ist, als die Eingabe.\\
In Zustand $q_e$ wird der Lesekopf zur\"uck nach links bewegt, damit die Ausgabe vollst\"andig ist.\\
Da ab dem zweitniedrigsten Bit mit 1 addiert wird, entspricht das Ergebnis der Addition mit 10, also dezimal 2.

\end{document}  